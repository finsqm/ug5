% this TeX file provides an awesome example of how TeX will make super 
% awesome tables, at the cost of your of what happens when you try to make a
% table that is very complicated.
% Originally turned in for Dr. Nico's Security Class
\documentclass[11pt]{article}


% Use wide margins, but not quite so wide as fullpage.sty
\marginparwidth 0.5in 
\oddsidemargin 0.25in 
\evensidemargin 0.25in 
\marginparsep 0.25in
\topmargin 0.25in 
\textwidth 6in \textheight 8 in
% That's about enough definitions

% multirow allows you to combine rows in columns
\usepackage{multirow}
% tabularx allows manual tweaking of column width
\usepackage{tabularx}
% longtable does better format for tables that span pages
\usepackage{longtable}
\usepackage{mathtools}

\begin{document}
% this is an alternate method of creating a title
%\hfill\vbox{\hbox{Gius, Mark}
%       \hbox{Cpe 456, Section 01}  
%       \hbox{Lab 1}    
%       \hbox{\today}}\par
%
%\bigskip
%\centerline{\Large\bf Lab 1: Security Audit}\par
%\bigskip
\author{Finlay McAfee - s1220880}
\title{Algorithmic Game Theory and Applications - Coursework 1}
\maketitle

\section{}
Let A be the bimatrix defining the 2-player strategic game G:
\begin{equation}
  A = 
  \begin{bmatrix}
    (6, 8) & (2,9) & (3,8) & (2,8) \\
    (0, 5) & (2,3) & (2,6) & (8,4) \\
    (7, 0) & (2,7) & (4,4) & (4,3) \\
    (2, 3) & (5,3) & (2,5) & (5,4) \\
  \end{bmatrix}
\end{equation}
\subsection{Elimination}
First we can eliminate the strongly dominated pure strategies from A to solve an equivalent but easier game.

For a pure strategy $x_i \in X_i$ to be strictly dominated there must exist another strategy $x'_i$ such that, for all $x_{-i} \in X_{-i}$:
\begin{equation}
  U_i(x_{-i};x'_i) > U_i(x_{-i};x_i)
\end{equation}

Note that this $x'_i$ does not need to be a pure strategy. There are no pure strategies that strictly dominate any others in this game. But there are mixed strategies that dominate pure strategies.

Take $x'_2$ to be $(0,\frac{1}{4},\frac{3}{4},0)$. This mixed strategy strictly dominates $\pi_{2,1}$ and $\pi_{2,4}$ as:

\begin{align}
  U_2(\pi_{1,1};x'_2) = 8.25 \\
  U_2(\pi_{1,2};x'_2) = 5.25 \\
  U_2(\pi_{1,3};x'_2) = 4.75 \\
  U_2(\pi_{1,4};x'_2) = 4.5
\end{align}

Hence we can eliminate columns 1 and 4 from A:

\begin{equation}
  A' = 
  \begin{bmatrix}
    (2,9) & (3,8) \\
    (2,3) & (2,6) \\
    (2,7) & (4,4) \\
    (5,3) & (2,5)
  \end{bmatrix}
\end{equation}

By the same method we can choose $x'_1$ to be $(0,0,\frac{2}{3},\frac{1}{3})$ which strictly dominates $\pi_{1,1}$ and $\pi_{1,2}$.

Therefore we can solve the following reduced matrix to find all Nash Equilibrium for the game G:
\begin{equation}
  A'' = 
  \begin{bmatrix}
    (2,7) & (4,4) \\
    (5,3) & (2,5)
  \end{bmatrix}
\end{equation}

\subsection{Computing NE}

It is clear by inspection that there are no pure Nash Equilibria. This is because, for every profile of pure strategies, there is always a player that is better off unilaterally switching to the other available pure strategy. By the definition of a Nash Equilibrium, this rules out pure NEs for the game G. By a similar argument there are no NE where one player plays a pure strategy and the other a mixed. However Nash's theorem tells us that a NE does exist, and as it must be a mixed NE, both players must play each strategy with positive probability. This means that each pure strategy, for both players, is a best response in a NE, by the Useful Corollary to Nash's Theorem.

Let $x^*_1(1)$, the probability of player 1 playing the strategy 1 in the NE, be $a$ and $x^*_1(2)$ be $1-a$. Let $x^*_2(1)$ be $b$ and $x^*_2(2)$ be $1-b$.

\begin{align}
  U^*_1(x^*_{-1};\pi_{1,1}) = U^*_1(x^*_{-1};\pi_{1,2}) \\
  2b + 4(1-b) = 5b + 2(1-b) \\
  4 - 2b = 2 + 3b \\
  b = \frac{2}{5}
\end{align}

Hence $x^*_2 = (\frac{2}{5}, \frac{3}{5})$, or $(0,\frac{2}{5}, \frac{3}{5}, 0)$ under the original game.

\begin{align}
  U^*_2(x^*_{-1};\pi_{2,1}) = U^*_2(x^*_{-1};\pi_{2,2}) \\
  7a + 3(1-a) = 4a + 5(1-a) \\
  3 + 4a = 5 - a \\
  a = \frac{2}{5}
\end{align}

Hence $x^*_1 = x^*_2$ (under the reduced game) and the profile of the NE for game G is 

\begin{equation}
  [(0,0,\frac{2}{5}, \frac{3}{5}), (0,\frac{2}{5}, \frac{3}{5}, 0)]
\end{equation}
As there was no loss of generality in these assumptions, it is clear that this is the only NE that exists for this game.

\section{}

The LP corresponding to the given 2-player zero sum game can be specified by:

\begin{flalign*}
  \textbf{maximise} \quad v \\
  \textbf{subject to:} \\
  v - (7x_1 + 2x_2 + 6x_3 + 5x_4 + 2x_5) \leq 0 \\
  v - (x_1 + 6x_2 + 3x_3 + 5x_4 + 8x_5) \leq 0 \\
  v - (6x_1 + 8x_2 + 8x_3 + 4x_4 + 2x_5) \leq 0 \\
  v - (4x_1 + 3x_2 + 3x_3 + 7x_4 + 8x_5) \leq 0 \\
  v - (2x_1 + 5x_2 + 4x_3 + 4x_4 + 9x_5) \leq 0 \\
  x_1 + x_2 + x_3 + x_4 + x_5 = 1 \\
  x_i \geq 0 \quad \text{for} \quad j = 1,...,5
\end{flalign*}

This was computed using the {\tt linprog} function in {\tt MATLAB} in the following manner:

Constraints 1,...,5 were encoded in the form $A\textbf{x} \leq \textbf{b}$ where:

\begin{align}
  \textbf{b} = \textbf{0} \\
  A = 
  \begin{bmatrix}
  1 & -7 & -2 & -6 & -5 & -2 \\
  1 & -1 & -6 & -3 & -5 & -8 \\
  1 & -6 & -8 & -8 & -4 & -2 \\
  1 & -4 & -3 & -3 & -7 & -8 \\
  1 & -2 & -5 & -4 & -4 & -9
  \end{bmatrix}
\end{align}

Constraint 6 was encoded in the form:

\begin{align}
  \text{beq} = 1 \\
  Aeq = [0, 1, 1, 1, 1, 1]
\end{align}

And Contraint 7 was encoded as lower bounds for the variables $x_1, ..., x_5$

The program produced the following results:

\begin{align}
  v = 4.8333 \\
  x_1 = 0 \\
  x_2 = 0 \\
  x_3 = 0.3333 \\
  x_4 = 0.5 \\
  x_5 = 0.1667
\end{align}

Hence the minimax value of this game is 4.8333 and the optimal strategy for player 1 is $x^*_1 = (0,0,0.3333,0.5,0.1667)$.

The variables of the dual of the linear program can be interpreted as the optimal strategy for player 2.

\begin{flalign*}
  \textbf{minimise} \quad u \\
  \textbf{subject to:} \\
  u - (7y_1 + 1y_2 + 6y_3 + 4y_4 + 2y_5) \leq 0 \\
  u - (2y_1 + 6y_2 + 8y_3 + 3y_4 + 5y_5) \leq 0 \\
  u - (6y_1 + 3y_2 + 8y_3 + 3y_4 + 4y_5) \leq 0 \\
  u - (5y_1 + 5y_2 + 4y_3 + 7y_4 + 4y_5) \leq 0 \\
  u - (2y_1 + 8y_2 + 2y_3 + 8y_4 + 9y_5) \leq 0 \\
  y_1 + y_2 + y_3 + y_4 + y_5 = 1 \\
  y_i \geq 0 \quad \text{for} \quad j = 1,...,5
\end{flalign*}

Computing this dual LP using {\tt MATLAB} gives the following values:

\begin{align}
  u = 4.8333 \\
  y_1 = 0.5556 \\
  y_2 = 0.2778 \\
  y_3 = 0 \\
  y_4 = 0 \\
  y_5 = 0.1667
\end{align}

Hence player 2's optimal strategy is $x^*_2 = (0.5556,0.2778,0,0,0.1667)$.

\section{}

\begin{equation}
  \begin{bmatrix}
    1 & -1 \\
    -1 & 1
  \end{bmatrix}
\end{equation}

\subsection*{(a)}

The unique NE of the Matching Pennies game is given by $x^* = [(0.5,0.5),(0.5,0.5)]$

\subsection*{(b)}

\end{document}
